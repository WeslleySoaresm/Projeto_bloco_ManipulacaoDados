

## 🧪 Exercícios com instruções detalhadas

### 1. 🔁 Inverter um dicionário simples  
**Objetivo:** Crie um dicionário `{1: 'a', 2: 'b', 3: 'a'}` e inverta os pares.  
**Use:**  
- `for chave, valor in dict.items()`  
- `if valor not in novo_dict`  
- `append()`  

**Desafio extra:** Agrupe todas as chaves que têm o mesmo valor.

---

### 2. 📁 Salvar o dicionário invertido em JSON  
**Objetivo:** Pegue o dicionário invertido do exercício 1 e salve em um arquivo `invertido.json`.  
**Use:**  
- `import json`  
- `with open(..., 'w') as f:`  
- `json.dump()`  

---

### 3. 📖 Ler um dicionário de um arquivo JSON e inverter  
**Objetivo:** Crie um arquivo `dados.json` com um dicionário simples e leia ele para inverter.  
**Use:**  
- `json.load()`  
- `open(..., 'r')`  
- Função `invert()`  

---

### 4. 🔄 Reverter o dicionário invertido  
**Objetivo:** Crie uma função `desinvert()` que recebe o dicionário invertido e reconstrói o original.  
**Use:**  
- `for chave, lista in dict.items()`  
- `for item in lista:`  
- `novo_dict[item] = chave`  

---

### 5. 🧠 Contar quantas vezes cada valor aparece  
**Objetivo:** Depois de inverter, conte quantas chaves estão em cada lista.  
**Use:**  
- `len(lista)`  
- `print()`  

---

### 6. 🧪 Testar com tipos diferentes  
**Objetivo:** Teste a função `invert()` com valores como números, strings, booleanos e tuplas.  
**Use:**  
- `print(type(valor))`  
- `try/except` para erros inesperados  

---

### 7. 🧩 Criar uma versão sem listas  
**Objetivo:** Crie uma função que inverte o dicionário, mas guarda **apenas uma chave por valor** (a última que aparecer).  
**Use:**  
- `novo_dict[valor] = chave`  
- Sem `append()`  

---

### 8. 🧪 Inverter dicionário de alunos e notas  
**Objetivo:** Dado `{“Ana”: 8.5, “Bruno”: 7.0, “Carlos”: 8.5}`, inverta para saber quais alunos têm a mesma nota.  
**Use:**  
- `invert()`  
- `print()`  
- `json.dump()` (opcional)

---

